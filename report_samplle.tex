\documentclass{jlreq}

\usepackage{ultimate_phys_report}

\begin{document}
\ultimatephysreport{物理学実験,2}{2023,1}{2023,8,25}{曇り,10,1234,99}{ヤバい実験}{22X1-001,山田 太郎}{チャルメラ次郎,ラーメン,}{1}
\section{実験の目的}
本実験では、「ヤバいことをする」ことを目的とする。
\section{理論}
$ax^2+bx+c=0$ の解は、
\[
    x=\frac{-b\pm\sqrt{b^2-4ac}}{2a}
\]
である。
\section{使用装置}
ヤバい人
\section{実験方法}
ヤバいことをしてもらう。
\section{実測値}
ヤバい。
\section{考察}
このことから、ヤバいことが出来ると考察できる。
\end{document}